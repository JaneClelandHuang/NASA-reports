\begin{table*}[!t]
\centering
%\resizebox{\textwidth}{!}
 \small\addtolength{\tabcolsep}{-4.8pt}
\caption{Key attributes that serve as indicators of various types of faults in ArduPilot and Px4}
\label{tab:Indicators}
\begin{tabular}{|L{.2cm}|L{6.4cm}|L{5cm}|L{4.5cm}|}
\hline
 & Description& ArduPilot & PX4 messages   \\ \hline
\multirow{5}{*}{\rotatebox[origin=c]{90}{\bf Mechanical}} & {\bf Actual Attitude}: represented by roll, pitch, and yaw in degrees & ATT.Roll, ATT.Pitch, ATT.Yaw  & vehicle\_attitude\_estimate \\ \cline{2-4}
 & {\bf Desired Attitude}: represented by roll, pitch, and yaw in degrees & ATT.DesRoll, ATT.DesPitch, ATT.DesYaw & vehicle\_attitude\_setpoint  \\ \cline{2-4} 
 & {\bf Actual Attitude Rates} to compensate for difference in actual and desired attitude (degrees/sec) & PIDR.Act PIDP.Act PIDY.Act RATE ( for some old versions)  & vehicle\_angular\_velocity \\ \cline{2-4}
 & {\bf Desired Attitude Rates} needed to compensate for difference in actual and desired attitude (degrees/sec) & PIDR.Tar PIDP.Tar PIDY.Tar  & vehicle\_rates\_setpoint \\ \cline{2-4}
 & {\bf PWM outputs} to power ESCs/motors (Typically set between 1100 to 1900)& RCOUT & actuator\_outputs  \\ \hline
 
 \multirow{3}{*}{\rotatebox[origin=c]{90}{\bf Vibration}}  & {\bf Vibration measures}: Standard deviations of accelerometer measurements (typically values under 30 are acceptable) & VIBE.VibeX, VIBE.VibeY, VIBE.VibeZ, VIBRATION.vibration\{\_x,\_y\_z\}$^T$   & N/A  \\  \cline{2-4}
 & {\bf Raw accelerometer values}: PID outputs from rate controller sent to the mixer to generate PWM outputs & IMU.AccX, IMU.AccY, IMU.AccZ  & vehicle\_acceleration \\ \cline{2-4}
  & Actuator controls & RCIN & actuator\_controls  \\ \hline

\multirow{3}{*}{\rotatebox[origin=c]{90}{{\hfill{ \bf Compass Error}\hfill}}}: & {\bf Throttle}: Shows generated throttle signal & CTUN.\{ThO,Thl\}, CURRENT.thr, RCIN.C3, VFR\_HUD.throttle, actuator\_ controls.control $^T$  & actuator\_controls.control  \\ \cline{2-4}
 & {\bf Raw magnetic field measurements}: across x,y, and z axes  & MAG.MagX, MAG.MagX, MAG.MagZ,  & vehicle\_magnetometer sensor\_mag  \\ \cline{2-4}
 & {\bf Norm of magnetic field}: considering raw values across x,y, \& z axes &  CUSTOM.mag\_field$^T$  & N/A   \\ \hline
 
 \multirow{3}{*}{\rotatebox[origin=c]{90}{\bf Power}} & {\bf Battery Voltage} & BAT.Volt CURR.Volt  & battery\_status.voltage\_v  \\ \cline{2-4} 
 & {\bf Board Voltage}: received at the board & POWR.Vcc, CURR.Vcc, HWSTATUS.Vcc$^T$    & N/A  \\ \cline{2-4}
 & {\bf Battery Current}: drawn from the battery & BAT.Curr CURR.Curr  & battery\_status.current\_c \\ \hline
 
\multirow{3}{*}{\rotatebox[origin=c]{90}{\bf GPS}}&
{\bf Satellites}: Number of satellites visible to receiver
 & GPS.NSats & sensor\_gps.satellites\_used satellite\_info.count, GPS\_RAW\_IT.satellites\_visible$^T$   \\ \cline{2-4}
&{\bf Dilution of Precision (DOP)}: Accuracy measure of GPS signal dependent upon the geometry of connected satellites  & GPS.HDop GPS.VDop  & sensor\_gps.hdop sensor\_gps.vdop, GPS\_RAW\_IT.eph$^T$    \\  \cline{2-4}
 & {\bf Position accuracy}: Standard deviation of horizontal and vertical position error & N/A & sensor\_gps.eph sensor\_gps.epv   \\ \hline

\end{tabular}%
\\ \vspace{2pt}
{\bf Attribute$^T$}: Attributes sent to ground control station and available in Telemetry log (TLOG). All other attributes stored onboard.

\end{table*}
\section{Detect and Diagnose Abnormal Flight Patterns:} 
The aim of these activities is to train and evaluate a time-series predictive analytics solution for automatically detecting, labeling, and classifying flight anomolies. These activities are largely dependent upon infrastructure-building tasks T1; however, we have started work on Tasks T2.1a and T2.2a as discussed below.

\begin{itemize}[leftmargin=0em]

\item[] 

\begin{itemize}[leftmargin=*]
\item {\bf Task 2.1:}
Create flight anomaly taxonomy\vspace{-8pt}
\begin{table}[h!]
\addtolength{\tabcolsep}{-5.6pt}

\hspace*{.38cm}\begin{tabular}{L{11.1cm}L{1.5cm} L{2cm} L{1.4cm} }
{\bf \scriptsize \sc Subtask}&{\bf \scriptsize \sc Started}&{\bf \scriptsize \sc Target}&{\bf \scriptsize \sc Status}\\ \hline
\sethlcolor{ylw}
\large a.~\hl{Create an initial taxonomy of common sUAS anomalies}&05/21&10/21&Completed\\
\large b.~{Extend the taxonomy with additional anomaly patterns}&&&Planned\\
% Add more if you want
\end{tabular}
\end{table}\vspace{-4pt}
\end{itemize}

\begin{itemize}[leftmargin=*]
\item {\bf Task 2.2:}
Train and test predictive analytics and anomoly classifier\vspace{-8pt}
\begin{table}[h!]
\addtolength{\tabcolsep}{-5.6pt}

\hspace*{.38cm}\begin{tabular}{L{11.1cm}L{1.5cm} L{2cm} L{1.4cm} }
{\bf \scriptsize \sc Subtask}&{\bf \scriptsize \sc Started}&{\bf \scriptsize \sc Target}&{\bf \scriptsize \sc Status}\\ \hline
\sethlcolor{ylw}
\large a.~\hl{Proof-of-concept using LSTM with Ardupilot \& px4 logs}&05/21&10/21&Active\\
\large b.~Train \& evaluate MSCRED using simulated data&&&Planned\\
\large c.~Train \& evaluate MSCRED on physical data&&&Planned\\
\large d.~Minimize the model for on-board deployment&&&Planned\\
\large e.~Train classifier to generate human-understandable tags &&&Planned\\
% Add more if you want
\end{tabular}
\end{table}\vspace{-4pt}
\end{itemize}
\end{itemize}

\begin{figure*}[t]

  \begin{subfigure}{0.5\textwidth}
    \centering
    \includegraphics[width=.8\textwidth]{figures/mag_detection.png}
    \caption{Detection of anomalous compass interference}
    \label{fig:res1}
  \end{subfigure}%
  \begin{subfigure}{0.5\textwidth}
    \centering
    \includegraphics[width=0.8\textwidth]{figures/ADPD-003__detection.jpg}
    \caption{Detection on the difference between desired and actual roll}
    \label{fig:res2}
  \end{subfigure}
  
  \begin{subfigure}{0.5\textwidth}
    \centering
    \includegraphics[width=.8\textwidth]{figures/ADPD-048__detection_y.jpg}
    \caption{Detection of anomalous vibration}
    \label{fig:res3}
  \end{subfigure}%
  \begin{subfigure}{0.5\textwidth}
    \centering
    \includegraphics[width=0.8\textwidth]{figures/ADPD-020__detection.jpg}
    \caption{Detection of anomalous vibration which does not exceed the documented threshold of 30 ms$^{-2}$}
    \label{fig:res4}
  \end{subfigure}
  
  \caption{Anomaly detections by the LSTM models}
  \label{fig:res}
\end{figure*}
\begin{table}[]
\centering
\small\addtolength{\tabcolsep}{-2.4pt}
%\small\addtolength{\tabcolsep}{-1pt} %Add this back if we need it.
\caption{\centering Performance of the models on test set using maximum errors on training data as thresholds}
    \label{tab:eval}
\begin{tabular}{|l|l|l|l|l|l|l|l|l|}
\hline

&\multirow{2}{*}{\textbf{Metrics}}
& \multicolumn{3}{c|}{\textbf{Vibration}}
& \multicolumn{3}{c|}{\textbf{Attitude}}  
& \textbf{Int} \\ \cline{3-9}

&& \textbf{VibeX} &\textbf{VibeY}& \textbf{VibeZ} & \textbf{Roll} & \textbf{Pitch} & \textbf{Yaw} & \textbf{Mag} \\ \hline

\multirow{4}{*}{\rotatebox{90}{\textbf{LSTM}}}
&Precision&0.92&0.85&0.95&1.00&0.90&0.78&0.83\\ \cline{2-9}
&Recall&0.76&0.89&0.86&0.88&0.64&0.64&1.00\\ \cline{2-9}
&Accuracy&0.89&0.89&0.91&0.95&0.87&0.86&0.90\\ \cline{2-9}
&F1&0.83&0.87&0.90&0.94&0.75&0.70&0.91\\ \cline{1-9}
%&Avg time/log (sec)&N/A&0.941&2.272&1.064&1.711&&\\ \cline{1-9}


\multirow{4}{*}{\rotatebox{90}{\textbf{ANN}}}
&Precision&0.88&0.80&0.87&0.94&0.75&0.75&1.00\\ \cline{2-9}
&Recall&0.82&0.84&0.91&1.00&0.85&0.56&0.60\\ \cline{2-9}
&Accuracy&0.89&0.85&0.89&0.97&0.87&0.84&0.80\\ \cline{2-9}
&F1&0.85&0.82&0.89&0.97&0.80&0.63&0.75\\ \cline{1-9}

\multirow{4}{*}{\rotatebox{90}{\textbf{Rule}}}

&Precision&1.00&1.00&1.00&n/a&n/a&n/a&n/a\\ \cline{2-9}
&Recall&0.18&0.26&0.77&n/a&n/a&n/a&n/a\\ \cline{2-9}
&Accuracy&0.70&0.70&0.90&n/a&n/a&n/a&n/a\\ \cline{2-9}
&F1&0.30&0.42&0.87&n/a&n/a&n/a&n/a\\ \cline{1-9}


\end{tabular}

\end{table}
Specific activities related to anomaly detection and diagnostics performed during this period have focused on the following:
\begin{itemize}
    \item {\bf Taxonomy of known sUAS Anomalies: {\it [Task 2.1a]} } We reopened this task in order to extend the taxonomy to include specific attributes for anomaly types. Documenting these attributes was necessary for building anomaly detectors. Results are shown in Table \ref{tab:Indicators}.
    \item {\bf Detecting anomalies using Autoencoders: {\it [Task 2.2a]} }
    We extended our previous experiments to build autoencoders (using LSTM and ANN) for three types of failures -- namely Vibration, Attitude, and Compass Interference, and conducted controlled experiments to evaluate effectiveness.   We also compared a heuristic approach for Attitude; however, this underperformed. The Compass interference was an entirely new, more complex time-series, as it included the interplay of two different attributes.
    

    Time-series plots showing detected anomalies for all three anomaly types are shown in \ref{fig:res2}. Experimental results from our experiments are reported in Table \ref{tab:eval}. These results are only intended as a baseline. We discuss shortcomings and next steps in the future work section. Our findings showed that ANN worked *almost* as well as LSTM and was lighter-weight and faster for the two simpler failure types -- Vibration and Attitude; however, Compass Interference, which was detected based on the interplay between throttle and MAG, performed better with LSTM. Overall results were very promising for an initial baseline.
    
    Our paper was submitted to the A* ranked International Conference on Software Processing and Networks. 
   
\end{itemize}
\subsection*{Next Steps: }
Next steps will focus on two aspects of the project.

\begin{itemize}
    \item {\bf Multivariate Analysis:}
While our work established a good baseline and served as a proof-of-concept that it is possible to detect emergent anomalous behavior, it has several shortcomings that we plan to address.  First, we trained individual autoencoders for each anomaly type; however, in reality, individual attributes serve as indicators across multiple types of failures. We therefore plan to look at multi-variate approaches that are capable of analysing multiple `symptoms' simultaneously and diagnosing diverse failures. This next step is a multi-month effort as it requires significant effort to build a far more extensive labeled dataset and also conduct extensive experiments. This part of the work will be primarily supervised by Dr. Chawla, with PhD students Yihong and Doheon (new student). The data collection will be a full-team effort.

\item {\bf Realtime Analysis:} Current experiments were all conducted on stored flight logs. However, in the next phase we will focus on real-time analysis. We want to tackle this early so that we fully understand performance and timing constraints.  In the next period we will start integrating anomaly detection on our autopilot flight logs (PX4, Ardupilot). We plan to get this working within a MAPE-K loop (Monitor-Analyze-Plan-Execute over a shared Knowledge) hosted onboard our autopilot. We plan to establish a test environment that includes (1) injecting flight errors into the simulator, (2) detecting errors, and (3) taking basic remedial actions.  Flight errors will be created in two ways -- first, using settings in the simulator which are designed to test for failure cases, and second, by forcing harmful configurations during flight (i.e, Myra's ISU team outputs).  We have an aggressive plan to submit a paper to SEAMS: Software Engineering for Adaptive and Self-Managing Systems Symposium with a January 20th submission deadline. SEAMS is a highly-focused, well-respected event with high relevance for our work.  As a contingency plan; if we are not able to complete experiments in time, we will target the March 10th deadline of the ACM Joint European Software Engineering Conference and Symposium on the Foundations of Software Engineering (A* conference).  The onboard infrastructure will be created by Nafee (ND), and the overall project will be a collaborative effort between Cleland-Huang/ND team and Cohen/ISU team.
\end{itemize}

 


