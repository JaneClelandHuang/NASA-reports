\documentclass[letterpaper]{article}

%%%%%%%%%%%%%%%%%%%%%%%%%%%%%
% Packages
%%%%%%%%%%%%%%%%%%%%%%%%%%%%%
\usepackage{enumitem}
\usepackage{multirow}
\usepackage{rotating}
\usepackage{color}
\usepackage{verbatim}
\usepackage{url}
\usepackage{afterpage}
\usepackage{wrapfig}
\usepackage{caption}
\usepackage{subcaption}
\usepackage{graphicx}
\usepackage{titlesec}
\usepackage[font={footnotesize,bf}]{caption}
\usepackage[table,xcdraw]{xcolor}
\usepackage{siunitx}
\usepackage{booktabs}
\usepackage{amsmath}
\usepackage{amssymb}
\usepackage{graphicx}
\usepackage{soul}
%myra added package to make citations go in numerial order
\usepackage{cite}
\usepackage[ruled,vlined]{algorithm2e}
\usepackage{algpseudocode}
\newcommand*{\mybox}[2]{\colorbox{#1!30}{\parbox{.98\linewidth}{#2}}}

\usepackage{scalerel}
\usepackage{xcolor, soul}


%%%%%%%%%%%%%%%%%%%%%%%%%%%%%%
% Commands
%%%%%%%%%%%%%%%%%%%%%%%%%%%%%%
\newcommand{\hilight}[1]{\colorbox{yellow}{#1}}
\setlength{\parskip}{1ex}
\usepackage{flushend}
\usepackage{xcolor}
\definecolor{dkgreen}{rgb}{0,0.6,0}
\definecolor{grayhighlight}{rgb}{.7,0.8,0.8}
\definecolor{col}{HTML}{b3d9ff}%{336699}
\definecolor{col2}{HTML}{A9A9A9}
\definecolor{ylw}{HTML}{e4d96f}
\sethlcolor{col}
 
% Change these to your own initials and colors for notes
\newcommand{\jch}[1]{{\textcolor{purple}{#1}}}
\newcommand{\mc}[1]{{\textcolor{blue}{#1}}}  %Myra
\newcommand{\nc}[1]{{\textcolor{dkgreen}{#1}}}  %Nitesh
\newcommand{\memo}[1]{\textbf{\textcolor{orange}{#1}}}
\newcommand{\task}[1]{{\scriptsize \bf{(#1)}}}
\newcommand{\chapfnt}{\fontsize{16}{19}}
\newcommand{\secfnt}{\fontsize{12}{13}}
\newcommand{\ssecfnt}{\fontsize{12}{14}}




%%%%%%%%%%%%%%%%%%%%%%%%%%%%%%%%%%%%%%%%%%%%%%%%%%%%%%%%%%%%%%%%%%%%%%%%%
\pagestyle{plain}  %%
%%%%%%%%%% EXACT 1in MARGINS %%%%%%%                                   %%
\setlength{\textwidth}{6.5in}     %%                                   %%
\setlength{\oddsidemargin}{0in}   %% (It is recommended that you       %%
\setlength{\evensidemargin}{0in}  %%  not change these parameters,     %%
\setlength{\textheight}{9.0in}    %%  at the risk of having your       %%
\setlength{\topmargin}{0in}       %%  proposal dismissed on the basis  %%
\setlength{\headheight}{0in}      %%  of incorrect formatting!!!)      %%
\setlength{\headsep}{0in}         %%                                   %%
\setlength{\footskip}{.3in}       %%     
\fontsize{12}{13}
\selectfont %%
%%%%%%%%%%%%%%%%%%%%%%%%%%%%%%%%%%%%                                   %%
%\newcommand{\required}[1]{\section*{\hfil #1\hfil}}                    %%
\renewcommand{\refname}{\hfil References Cited\hfil}  
%%%%%%%%%%%%%%%%%%%%%%%%%%%%%%%%%%%%%%%%%%%%%%%%%%%%%%%%%%%%%%%%%%%%%%%%

\makeatletter
\DeclareTextFontCommand{\emph}{\bfseries}{
	\@nomath\em \if b\expandafter\@car\f@series\@nil
	\normalfont \else \bfseries \fi}
\newcolumntype{L}[1]{>{\raggedright\let\newline\\\arraybackslash}p{#1}}  %\hspace{0pt}\vspace{4pt}
\newcolumntype{C}[1]{>{\centering\let\newline\\\arraybackslash}p{#1}} 
\newcommand{\boxnum}[2][table-format=2.1]{\smash{\fbox{\vphantom{#2}\tablenum[#1]{#2}}}}
\def\@makechapterhead#1{%
  \vspace*{50\p@}%
  {\parindent \z@ \raggedright \normalfont
    \ifnum \c@secnumdepth >\m@ne
        \chapfnt\bfseries \@chapapp\space \thechapter
        \par\nobreak
        \vskip 20\p@
    \fi
    \interlinepenalty\@M
    \chapfnt \bfseries #1\par\nobreak
    \vskip 40\p@
  }}
\renewcommand\section{\@startsection {section}{1}{\z@}{8pt}{4pt} {\large\secfnt\bfseries}}
\renewcommand\subsection{\@startsection{subsection}{2}{\z@}{2pt}{2pt}%
{\normalfont\ssecfnt\bfseries}}
\makeatother



\usepackage{hyperref}
\usepackage{array}
\usepackage{makecell}

\hypersetup{
    colorlinks=true,
    linkcolor=blue,
    filecolor=magenta,      
    urlcolor=blue,
    pdftitle={Overleaf Example},
    pdfpagemode=FullScreen,
    }
\begin{document}
\fontsize{12}{13}
\selectfont

\begin{center}
\textbf{\Large{\bf Safe Deployment of Small Unmanned Aerial Systems\\}}
\vspace{6pt}
\textbf{\Large{\bf through On-Board Monitoring and Assessment}}\\ 
\vspace{6pt}
\textbf{NRA ROA –2019 NNH19ZEA001N-SWS\\
B.5 System-Wide Safety (SWS) Project}\\ \vspace{6pt}
\underline{\smash{\emph{Monthly Report: December 2021}}} \\ \vspace{6pt}
PI: Jane Cleland-Huang, (JaneHuang@nd.edu)\\
co-I: Myra Cohen, (mcohen@iastate.edu)\\
co-I: Nitesh Chawla,(nchawla@nd.edu)\\
\end{center}

\section{General Overview}
As reported in our last month's report, we recently submitted a paper to the International Conference on Information Processing in Sensor Networks (IPSN), entitled: {\bf Md.Nafee Al Islam, Yihong Ma, Nitesh Chawla, Jane Cleland-Huang, `Detecting and Diagnosing Anomalies in the Sensor Data of small Unmanned Aerial Systems'.} As a result, the past few weeks have marked a transition in which we have focused on conceptual planning of next steps for the failure analysis algorithms, and preliminary work for real-time onboard analytics. In addition, the ISU team has focused upon creating the curated px4 dataset and building their experimental infrastructure, whilst the ND team has continued efforts in developing the annotation website. 

We first report on specific tasks, and then in Section \ref{sec:response} answer Steven Young's questions from our 6 Month review presentation.

\section{Data Collection and Annotation Infrastructure}
Task T1 focuses on establishing infrastructure for the project. 
Our goal is to have much of the basic infrastructure developed by the end of November. the infrastructure for uploading files and annotating them fully deployed by the end of November (Task 1.1a). To that end we have engaged in the following infrastucture building tasks:

\sethlcolor{col}

\begin{itemize}[leftmargin=0em]

\item[] 

\begin{itemize}[leftmargin=*]
\item {\bf Task 1.1:}
Develop data collection and annotation infrastructure.\vspace{-10pt}
\begin{table}[h!]
\addtolength{\tabcolsep}{-6pt}
%\small
\hspace*{.38cm}\begin{tabular}{L{11.1cm}L{1.5cm} L{2cm} L{1.4cm} }
{\bf \scriptsize \sc Subtask}&{\bf \scriptsize \sc Started}&{\bf \scriptsize \sc Target}&{\bf \scriptsize \sc Status}\\ \hline

\large a.\hl{~Develop \& deploy website for collecting flight logs}.&06/21&11/21&Active\\
\sethlcolor{col2}
\large b.\hl{~Develop tools for visualizing \& analyzing flight anomalies}.&06/21&11/21&Active\\
\sethlcolor{col2}
\large c.\hl{~Build collections of anomalous flight logs}&07/21&Continual&Active\\
\large d.~Construct labeled dataset of flight log anomalies. &&&Planned\\
\end{tabular}
\end{table}\vspace{-8pt}

\item{\bf Task 1.2:} Create infrastructure for working with simulated data. \vspace{-10pt}
\normalsize
\begin{table}[h!]
\addtolength{\tabcolsep}{-5.6pt}
\normalsize
\hspace*{.38cm}\begin{tabular}{L{11.1cm}L{1.5cm} L{2cm} L{1.4cm} }
{\bf \scriptsize \sc Subtask}&{\bf \scriptsize \sc Started}&{\bf \scriptsize \sc Target}&{\bf \scriptsize \sc Status}\\ \hline
\sethlcolor{col}
\large a.~\hl{Establish environment for collecting simulated flight logs}&06/21&09/21&Active\\
\large b.~Collect paired flight logs for simulated \& physical flights.&&&Planned\\
\large c.~Establish test oracle \& statistically compare flight data.&&&Planned\\
\large d.~Create cloud service for runtime off-board data analysis &&&Planned\\
\end{tabular}
\end{table}
\end{itemize}
\end{itemize}
\vspace{-12pt}

Specific activities related to data collection infrastructure that we have performed during this period have focused on the following:
\begin{itemize}[leftmargin=*]
    \item {\bf Backend Server for Data Collection} {\it  [Task T1.1a]: } We have had several discussions with our Computing Research Center (CRC) who were originally going to deploy the backend database for us.  Due to some internal CRC circumstances they are no longer able to do this.  Instead we will develop the full website ourselves and host on a cloud-based service.  We have just recruited a Senior UG student (Raj Mehta) who will deploy an initial Django website and then connect the front end UI. While we experienced a delay on this task, we still anticipate meeting our scheduled deployment.
    \item {\bf Front-end Data Collection website: } The previous front-end work was a summer project with two UG students. We do not plan further activities until the backend server (at least an initial version) is deployed.
    \item {\bf Initial Data Collation} {\it  [Task 1.1b]: }
    There were no further collection activities this month as we are currently using the Ardupilot and px4 dataset that we have already downloaded (px4) and collected (ardupilot).
\item{\bf Onboard Infrastructure} {\it [Task 1.2a]: }
We have continued work on developing the onboard infrastructure and a large-scale, robust, realistic simulation environment for ecosystem experimentation.  Once released, each simulation will support up to 230 simultaneous UAVs (each with a full physics engine simulation and onboard camera) across any region. We will report on this further once it is fully deployed and operational (hopefully next month). While we have previously deployed simulations with 200+ drones, the new simulation environment is needed as it introduces greater fidelity, interactivity between the drones, and an integrated environment for supporting both onboard flight log analysis as well as off-board data aggregation and analysis. We have addressed several challenging technical problem but we are making steady progress.  Once deployed, this will speed up our work significantly and enable experiments at scales and degrees of fidelity not feasible with physical drones alone.  \end{itemize}

\subsection*{Summarized Metrics}
As building a large, annotated dataset is a key goal of the project, we track progress towards this goal. We did not extend this part of the dataset since the last report; however, Cohen's team are working on a subset of same dataset (px4) and ultimately we will merge results into a single dataset.

\begin{table}[h!]
    \centering 
    \begin{tabular}{|l|l|c|c|c|} \hline
        &{\bf Metric} & {\bf PX4} & {\bf Ardupilot} & {\bf Other}  \\ \hline
        1.& Total No. of Physical Logs collected&77357&700&0\\ \hline
        2.& Total No. of Physical Logs collected with anomalies&tbd&$>$30&0\\ \hline 
        3.& Total No. of Physical Logs collected with annotated anomalies&0 &0&0\\ \hline  
        4.& Total No. of Simulated Logs collected&0&0&0\\ \hline
        5.& Total No. of Simulated Logs collected with anomalies&0 &0&0\\ \hline 
        6.& Total No. of Simulated Logs collected with anomalies labeled&0 &0&0\\ \hline  
    \end{tabular}
    \newline Note: Still analyzing all the data we imported as it lacks some meta-data.
\end{table}

\subsection*{Next Steps: }
The majority of effort next month will go to infrastructure of the backend database, the onboard collection mechanisms, and the large-scale simulation.
\begin{table*}[!t]
\centering
%\resizebox{\textwidth}{!}
 \small\addtolength{\tabcolsep}{-4.8pt}
\caption{Key attributes that serve as indicators of various types of faults in ArduPilot and Px4}
\label{tab:Indicators}
\begin{tabular}{|L{.2cm}|L{6.4cm}|L{5cm}|L{4.5cm}|}
\hline
 & Description& ArduPilot & PX4 messages   \\ \hline
\multirow{5}{*}{\rotatebox[origin=c]{90}{\bf Mechanical}} & {\bf Actual Attitude}: represented by roll, pitch, and yaw in degrees & ATT.Roll, ATT.Pitch, ATT.Yaw  & vehicle\_attitude\_estimate \\ \cline{2-4}
 & {\bf Desired Attitude}: represented by roll, pitch, and yaw in degrees & ATT.DesRoll, ATT.DesPitch, ATT.DesYaw & vehicle\_attitude\_setpoint  \\ \cline{2-4} 
 & {\bf Actual Attitude Rates} to compensate for difference in actual and desired attitude (degrees/sec) & PIDR.Act PIDP.Act PIDY.Act RATE ( for some old versions)  & vehicle\_angular\_velocity \\ \cline{2-4}
 & {\bf Desired Attitude Rates} needed to compensate for difference in actual and desired attitude (degrees/sec) & PIDR.Tar PIDP.Tar PIDY.Tar  & vehicle\_rates\_setpoint \\ \cline{2-4}
 & {\bf PWM outputs} to power ESCs/motors (Typically set between 1100 to 1900)& RCOUT & actuator\_outputs  \\ \hline
 
 \multirow{3}{*}{\rotatebox[origin=c]{90}{\bf Vibration}}  & {\bf Vibration measures}: Standard deviations of accelerometer measurements (typically values under 30 are acceptable) & VIBE.VibeX, VIBE.VibeY, VIBE.VibeZ, VIBRATION.vibration\{\_x,\_y\_z\}$^T$   & N/A  \\  \cline{2-4}
 & {\bf Raw accelerometer values}: PID outputs from rate controller sent to the mixer to generate PWM outputs & IMU.AccX, IMU.AccY, IMU.AccZ  & vehicle\_acceleration \\ \cline{2-4}
  & Actuator controls & RCIN & actuator\_controls  \\ \hline

\multirow{3}{*}{\rotatebox[origin=c]{90}{{\hfill{ \bf Compass Error}\hfill}}}: & {\bf Throttle}: Shows generated throttle signal & CTUN.\{ThO,Thl\}, CURRENT.thr, RCIN.C3, VFR\_HUD.throttle, actuator\_ controls.control $^T$  & actuator\_controls.control  \\ \cline{2-4}
 & {\bf Raw magnetic field measurements}: across x,y, and z axes  & MAG.MagX, MAG.MagX, MAG.MagZ,  & vehicle\_magnetometer sensor\_mag  \\ \cline{2-4}
 & {\bf Norm of magnetic field}: considering raw values across x,y, \& z axes &  CUSTOM.mag\_field$^T$  & N/A   \\ \hline
 
 \multirow{3}{*}{\rotatebox[origin=c]{90}{\bf Power}} & {\bf Battery Voltage} & BAT.Volt CURR.Volt  & battery\_status.voltage\_v  \\ \cline{2-4} 
 & {\bf Board Voltage}: received at the board & POWR.Vcc, CURR.Vcc, HWSTATUS.Vcc$^T$    & N/A  \\ \cline{2-4}
 & {\bf Battery Current}: drawn from the battery & BAT.Curr CURR.Curr  & battery\_status.current\_c \\ \hline
 
\multirow{3}{*}{\rotatebox[origin=c]{90}{\bf GPS}}&
{\bf Satellites}: Number of satellites visible to receiver
 & GPS.NSats & sensor\_gps.satellites\_used satellite\_info.count, GPS\_RAW\_IT.satellites\_visible$^T$   \\ \cline{2-4}
&{\bf Dilution of Precision (DOP)}: Accuracy measure of GPS signal dependent upon the geometry of connected satellites  & GPS.HDop GPS.VDop  & sensor\_gps.hdop sensor\_gps.vdop, GPS\_RAW\_IT.eph$^T$    \\  \cline{2-4}
 & {\bf Position accuracy}: Standard deviation of horizontal and vertical position error & N/A & sensor\_gps.eph sensor\_gps.epv   \\ \hline

\end{tabular}%
\\ \vspace{2pt}
{\bf Attribute$^T$}: Attributes sent to ground control station and available in Telemetry log (TLOG). All other attributes stored onboard.

\end{table*}
\section{Detect and Diagnose Abnormal Flight Patterns:} 
The aim of these activities is to train and evaluate a time-series predictive analytics solution for automatically detecting, labeling, and classifying flight anomolies. These activities are largely dependent upon infrastructure-building tasks T1; however, we have started work on Tasks T2.1a and T2.2a as discussed below.

\begin{itemize}[leftmargin=0em]

\item[] 

\begin{itemize}[leftmargin=*]
\item {\bf Task 2.1:}
Create flight anomaly taxonomy\vspace{-8pt}
\begin{table}[h!]
\addtolength{\tabcolsep}{-5.6pt}

\hspace*{.38cm}\begin{tabular}{L{11.1cm}L{1.5cm} L{2cm} L{1.4cm} }
{\bf \scriptsize \sc Subtask}&{\bf \scriptsize \sc Started}&{\bf \scriptsize \sc Target}&{\bf \scriptsize \sc Status}\\ \hline
\sethlcolor{ylw}
\large a.~\hl{Create an initial taxonomy of common sUAS anomalies}&05/21&10/21&Completed\\
\large b.~{Extend the taxonomy with additional anomaly patterns}&&&Planned\\
% Add more if you want
\end{tabular}
\end{table}\vspace{-4pt}
\end{itemize}

\begin{itemize}[leftmargin=*]
\item {\bf Task 2.2:}
Train and test predictive analytics and anomoly classifier\vspace{-8pt}
\begin{table}[h!]
\addtolength{\tabcolsep}{-5.6pt}

\hspace*{.38cm}\begin{tabular}{L{11.1cm}L{1.5cm} L{2cm} L{1.4cm} }
{\bf \scriptsize \sc Subtask}&{\bf \scriptsize \sc Started}&{\bf \scriptsize \sc Target}&{\bf \scriptsize \sc Status}\\ \hline
\sethlcolor{ylw}
\large a.~\hl{Proof-of-concept using LSTM with Ardupilot \& px4 logs}&05/21&10/21&Active\\
\large b.~Train \& evaluate MSCRED using simulated data&&&Planned\\
\large c.~Train \& evaluate MSCRED on physical data&&&Planned\\
\large d.~Minimize the model for on-board deployment&&&Planned\\
\large e.~Train classifier to generate human-understandable tags &&&Planned\\
% Add more if you want
\end{tabular}
\end{table}\vspace{-4pt}
\end{itemize}
\end{itemize}

\begin{figure*}[t]

  \begin{subfigure}{0.5\textwidth}
    \centering
    \includegraphics[width=.8\textwidth]{figures/mag_detection.png}
    \caption{Detection of anomalous compass interference}
    \label{fig:res1}
  \end{subfigure}%
  \begin{subfigure}{0.5\textwidth}
    \centering
    \includegraphics[width=0.8\textwidth]{figures/ADPD-003__detection.jpg}
    \caption{Detection on the difference between desired and actual roll}
    \label{fig:res2}
  \end{subfigure}
  
  \begin{subfigure}{0.5\textwidth}
    \centering
    \includegraphics[width=.8\textwidth]{figures/ADPD-048__detection_y.jpg}
    \caption{Detection of anomalous vibration}
    \label{fig:res3}
  \end{subfigure}%
  \begin{subfigure}{0.5\textwidth}
    \centering
    \includegraphics[width=0.8\textwidth]{figures/ADPD-020__detection.jpg}
    \caption{Detection of anomalous vibration which does not exceed the documented threshold of 30 ms$^{-2}$}
    \label{fig:res4}
  \end{subfigure}
  
  \caption{Anomaly detections by the LSTM models}
  \label{fig:res}
\end{figure*}
\begin{table}[]
\centering
\small\addtolength{\tabcolsep}{-2.4pt}
%\small\addtolength{\tabcolsep}{-1pt} %Add this back if we need it.
\caption{\centering Performance of the models on test set using maximum errors on training data as thresholds}
    \label{tab:eval}
\begin{tabular}{|l|l|l|l|l|l|l|l|l|}
\hline

&\multirow{2}{*}{\textbf{Metrics}}
& \multicolumn{3}{c|}{\textbf{Vibration}}
& \multicolumn{3}{c|}{\textbf{Attitude}}  
& \textbf{Int} \\ \cline{3-9}

&& \textbf{VibeX} &\textbf{VibeY}& \textbf{VibeZ} & \textbf{Roll} & \textbf{Pitch} & \textbf{Yaw} & \textbf{Mag} \\ \hline

\multirow{4}{*}{\rotatebox{90}{\textbf{LSTM}}}
&Precision&0.92&0.85&0.95&1.00&0.90&0.78&0.83\\ \cline{2-9}
&Recall&0.76&0.89&0.86&0.88&0.64&0.64&1.00\\ \cline{2-9}
&Accuracy&0.89&0.89&0.91&0.95&0.87&0.86&0.90\\ \cline{2-9}
&F1&0.83&0.87&0.90&0.94&0.75&0.70&0.91\\ \cline{1-9}
%&Avg time/log (sec)&N/A&0.941&2.272&1.064&1.711&&\\ \cline{1-9}


\multirow{4}{*}{\rotatebox{90}{\textbf{ANN}}}
&Precision&0.88&0.80&0.87&0.94&0.75&0.75&1.00\\ \cline{2-9}
&Recall&0.82&0.84&0.91&1.00&0.85&0.56&0.60\\ \cline{2-9}
&Accuracy&0.89&0.85&0.89&0.97&0.87&0.84&0.80\\ \cline{2-9}
&F1&0.85&0.82&0.89&0.97&0.80&0.63&0.75\\ \cline{1-9}

\multirow{4}{*}{\rotatebox{90}{\textbf{Rule}}}

&Precision&1.00&1.00&1.00&n/a&n/a&n/a&n/a\\ \cline{2-9}
&Recall&0.18&0.26&0.77&n/a&n/a&n/a&n/a\\ \cline{2-9}
&Accuracy&0.70&0.70&0.90&n/a&n/a&n/a&n/a\\ \cline{2-9}
&F1&0.30&0.42&0.87&n/a&n/a&n/a&n/a\\ \cline{1-9}


\end{tabular}

\end{table}
Specific activities related to anomaly detection and diagnostics performed during this period have focused on the following:
\begin{itemize}
    \item {\bf Taxonomy of known sUAS Anomalies: {\it [Task 2.1a]} } We reopened this task in order to extend the taxonomy to include specific attributes for anomaly types. Documenting these attributes was necessary for building anomaly detectors. Results are shown in Table \ref{tab:Indicators}.
    \item {\bf Detecting anomalies using Autoencoders: {\it [Task 2.2a]} }
    We extended our previous experiments to build autoencoders (using LSTM and ANN) for three types of failures -- namely Vibration, Attitude, and Compass Interference, and conducted controlled experiments to evaluate effectiveness.   We also compared a heuristic approach for Attitude; however, this underperformed. The Compass interference was an entirely new, more complex time-series, as it included the interplay of two different attributes.
    

    Time-series plots showing detected anomalies for all three anomaly types are shown in \ref{fig:res2}. Experimental results from our experiments are reported in Table \ref{tab:eval}. These results are only intended as a baseline. We discuss shortcomings and next steps in the future work section. Our findings showed that ANN worked *almost* as well as LSTM and was lighter-weight and faster for the two simpler failure types -- Vibration and Attitude; however, Compass Interference, which was detected based on the interplay between throttle and MAG, performed better with LSTM. Overall results were very promising for an initial baseline.
    
    Our paper was submitted to the A* ranked International Conference on Software Processing and Networks. 
   
\end{itemize}
\subsection*{Next Steps: }
Next steps will focus on two aspects of the project.

\begin{itemize}
    \item {\bf Multivariate Analysis:}
While our work established a good baseline and served as a proof-of-concept that it is possible to detect emergent anomalous behavior, it has several shortcomings that we plan to address.  First, we trained individual autoencoders for each anomaly type; however, in reality, individual attributes serve as indicators across multiple types of failures. We therefore plan to look at multi-variate approaches that are capable of analysing multiple `symptoms' simultaneously and diagnosing diverse failures. This next step is a multi-month effort as it requires significant effort to build a far more extensive labeled dataset and also conduct extensive experiments. This part of the work will be primarily supervised by Dr. Chawla, with PhD students Yihong and Doheon (new student). The data collection will be a full-team effort.

\item {\bf Realtime Analysis:} Current experiments were all conducted on stored flight logs. However, in the next phase we will focus on real-time analysis. We want to tackle this early so that we fully understand performance and timing constraints.  In the next period we will start integrating anomaly detection on our autopilot flight logs (PX4, Ardupilot). We plan to get this working within a MAPE-K loop (Monitor-Analyze-Plan-Execute over a shared Knowledge) hosted onboard our autopilot. We plan to establish a test environment that includes (1) injecting flight errors into the simulator, (2) detecting errors, and (3) taking basic remedial actions.  Flight errors will be created in two ways -- first, using settings in the simulator which are designed to test for failure cases, and second, by forcing harmful configurations during flight (i.e, Myra's ISU team outputs).  We have an aggressive plan to submit a paper to SEAMS: Software Engineering for Adaptive and Self-Managing Systems Symposium with a January 20th submission deadline. SEAMS is a highly-focused, well-respected event with high relevance for our work.  As a contingency plan; if we are not able to complete experiments in time, we will target the March 10th deadline of the ACM Joint European Software Engineering Conference and Symposium on the Foundations of Software Engineering (A* conference).  The onboard infrastructure will be created by Nafee (ND), and the overall project will be a collaborative effort between Cleland-Huang/ND team and Cohen/ISU team.
\end{itemize}

 



\section{Fault detection and Feature Interactions in sUAS:} The aim of these tasks is to identify runaway emergent behaviors due to interactions between hardware, software and environmental conditions using feature-based modeling and testing and search-based exploration using simulation.  

\begin{itemize}[leftmargin=0em]
\item[] 
\begin{itemize}[leftmargin=*]
% First subtask
\item \textbf{Task 3.1:}~{Build and refine relational models of the sUAS configuration space.}
\vspace{-8pt}
\begin{table}[h!]
\addtolength{\tabcolsep}{-5.6pt}
\hspace*{.38cm}\begin{tabular}{L{11.1cm}L{1.5cm} L{2cm} L{1.4cm} }
{\bf \scriptsize \sc Subtask}&{\bf \scriptsize \sc Started}&{\bf \scriptsize \sc Target}&{\bf \scriptsize \sc Status}\\ \hline
\large a.~\hl{Build initial feature models} &06/21&10/21&Active\\
\large b.~\hl{Create feature-based simulation environment}   &08/21&12/21&Active\\
\large c.~Semantic partitioning of feature space  &&&Planned\\
\end{tabular}
\end{table}\vspace{-8pt}
% Second subtask
\item \textbf{Task 3.2:}~Systematically search for interaction Faults using  generated feature models. \vspace{-8pt}%Characterize features leading to failures as classification trees and utilize evolutionary algorithms for deeper search.  Experiments will be run in simulators on our HCC clusters. %Deliverables include the CIT samples, simulation scripts, evolutionary algorithms, experimental data and tagged artifacts of search. 
\begin{table}[h!]
\addtolength{\tabcolsep}{-5pt}

\hspace*{.38cm}\begin{tabular}{L{11.1cm}L{1.5cm} L{2cm} L{1.4cm} }
{\bf \scriptsize \sc Subtask}&{\bf \scriptsize \sc Started}&{\bf \scriptsize \sc Target}&{\bf \scriptsize \sc Status}\\ \hline
\normalsize
\large a.~CIT integrated and initial experiments run  &&&Planned\\
\large b.~Develop alternative CIT models \& use higher strength CIT  &&&Planned\\
\large c.~Evolutionary algorithms and integration with Goal 2 &&&Planned\\
% Add more if you want
\end{tabular}
\end{table}%\vspace{-4pt}
\end{itemize}
\end{itemize}\vspace{-16pt}


Specific activities and progress related to these tasks are:
\begin{enumerate}
\item Salil downloaded the first 15,142 public px4 logs found at (\url{https://logs.px4.io/browse)}. He analyzed the parameters across all logs to learn which ones are commonly changed from default values. Our reasoning is that it lets us build an initial feature model with features relevant to pilots. There are a total of 1,564 unique parameters appearing in these logs.  Figure \ref{fig:histogram} shows a histogram of this data using the default histogram settings in \texttt{R}.  The \texttt{y-axis} shows the frequency and the \texttt{x-axis} shows the number of times a parameter was changed; the histogram is bucketing parameters by how often they were different than the default value.  The breaks are at $1000, 2000, ..., 15,000$. 

First, we noticed that a small number of parameters (37 or 2.3\%) are modified in more than 90\% of the logs. We keep these as a special group. Since they are almost always modified some parameters in this group may be IDs or have other special meaning.  It is also likely that problems due to manipulation of these parameters may already be documented.  We also see that 39.7\% (or 621 parameters) always retain the default value. We ignore those for now.  Of those that are modified, 1,345 are modified less than 1,000 times and 943 (60.3\%) are modified less than 10 times.  We chose to start focusing our in-depth analysis in the middle, with parameters that seem to be relevant for pilots. We identified 94 parameters that are changed between 20\% and 77\% of the time.  We have read the documentation for this set of parameters, identified the possible value ranges, and any constraints (dependencies with other parameters) mentioned in the documentation.   

Figure \ref{fig:params} shows a screenshot of some of this information. It lists the parameter, the range of values, specific values (if listed in the documentation), the data type, a description and any documented constraints.  We have completed this initial step for the 94 parameters boxed in Figure \ref{fig:histogram} and are working on the 37 in the most frequently modified group now.

\begin{figure}[h]
\centering
\includegraphics[width=6.0in]{figures/parameters-histogram.pdf}
\caption{Histogram showing Frequency of Configuration Parameters which are changed from the non-default value from analysis of 15,142 public PX4 logs. 39.7\% are never modified. A small percentage (37 or 2.3\%) are modified in almost all logs (greater than 90\% of the time). We are focusing on those in the middle range (modified between 3,000 (20\%) and 11,164 (77.4\%) of the time, shown in the box.} \label{fig:histogram}
\end{figure}






\begin{figure}[h]
\centering
\includegraphics[width=6.0in]{figures/example-params.pdf}
\caption{Example parameters from the list of 94.} \label{fig:params}
\end{figure}



\item Urjoshi continues to work on an initial prototype for our genetic algorithms (Task 3.1c). She has set up JMetal ((\url{https://github.com/jMetal}), a genetic algorithm framework for an existing configuration project  (not related to drones) and has designed a chromosome for our genetic algorithm. Each gene of the chromosome will be a single parameter in the chromosome. For instance, if there are three parameters, there will be three genes and the different values for those parameters will make up the choices (or alleles) for those genes.  In the intiial px4 chromosome the length is 7. This type of simple chromosome using actual values (as opposed to a bit representation) tends to work well on similar other problems in configurability and since it maps directly to the problem space tuning is more straight forward. 

She has worked with Salil to identify the initial set of parameters to manipulate (see discussion above). She chose 7 parameters (6 from the group that are changed most of the time and 1 from the next group (the boxed area in Figure \ref{fig:histogram}). For  parameters in the first group she picked ones that seem to have a functional purpose (not ones with ID as part of their name).  Our initial feature model is seen in Figure \ref{fig:features}. This is a small part of the px4 parameter space, and is somewhat arbitrary, however it will allow us to explore an exhaustive space to test our algorithms, and is large enough to have an interesting landscape for search.  In addition it contains parameters that are modified often, meaning they should have relevance to pilots. Most of the features have 5-7 choices (the leaves), of which only one value can be selected at a time (they are mutually exclusive choices shown as a solid arc). The features are all required since at least one of their values must be selected (shown as solid circles).   One feature (IMU\_GYRO\_CAL\_EN) is a Boolean option that is either on or off (shown as optional with an open circle). This parameter enables IMU (Inertial Measurement Unit) gyro auto calibration.  

Urjoshi has also identified the startup files which change the various parameters in px4 this month, and has run some initial experiments to see if random changes to these parameters impact different log outputs. While we do not yet have a fitness defined, we will need to be able to differentiate data in the output logs, so we ran some tests to see if there is any variation in the outcomes. We first tested several repeated runs of these parameters using the (same) initial values to confirm that individual runs have only a small (if any) variation in their outputs. This suggests variation observed is due to changes in the parameters.  She next randomly selected 24 different configurations from the feature model. Table \ref{tab:variation} shows this data. It shows the minimum and maximum values for the log outputs: \textit{distance}, \textit{average speed}, \textit{max tilt} and \textit{roll rate range}. For \textit{roll rate range} we give the minimum, maximum and the range in our table. While we have not systematically explored this space, our data suggests that IMU\_GYRO\_RATEMAX has the largest impact on differences in these parameters. According to documentation this is \textit{the maximum rate the gyro control data (vehicle\_angular\_velocity) will be allowed to publish at. It is the loop rate for the rate controller and outputs}.

\begin{table}[h]
    \centering
     \caption{Variation in log values for several metrics over 24 different random configurations of the feature model. We show the minimum and maximum for each metric to get a sense for the size of variation. Individual simulations which are repeated using the same parameter values show almost not variation.}
    \label{tab:variation}
    \begin{tabular}{|l||r|r|}
    \hline
\textbf{Metric} & \textbf{Min} & \textbf{Max} \\
       \hline
       \hline
distance (km)	&1.0	&1.5 \\
       \hline

avg. speed (km/hr)&	12.1	&13.1 \\
       \hline

max tilt (degree)&	24.2&	26.4\\ 
       \hline

roll rate range (degree/sec)&	4.0&	8.0 \\
       \hline

roll rate min  (degree/sec) &	0.4 &	10.5\\
       \hline

roll rate max  (degree/sec)	& 4.5	& 14.0  \\
\hline         
    \end{tabular}
   
\end{table}

\begin{figure}[h]
\centering
\includegraphics[width=6.5in]{figures/featuremodel.pdf}
\caption{Initial Feature Model. Consists of 113,400 possible px4 configurations.} \label{fig:features}
\end{figure}


\end{enumerate}



\subsection*{Next Steps:}
For our next steps, Salil and Urjoshi are working with the Notre Dame team to learn more about the 131 starting parameters that are used at least 3,000 times in the log. They hope to pare this down to parameters which are relevant and to incorporate constraints (dependencies) between parameter groups.  Salil has now downloaded 53,020 logs and is re-analyzing the data to see if the histogram changes.  This data will also be used for additional mining of information.  Urjoshi is working on the infrastructure to automate changing the configurations, running the simulation and finally extracting data from the resultant logs. This is needed to fully automate our process.  Last, we have an undergraduate researcher who has joined our team. He is working to map the meta-data on the px4 log website to output files from the log parser. He is building a cross reference that our entire team can use  since much of the time series data is contained in different, external csv files after extraction. 



\section{Other Goals}
\label{sec:Goals}
\noindent Work on other research goals is either not yet started or in very preliminary stages.  There are no changes in this work yet.
\subsection{Safety Analysis and Assurance: }
Our  goal  is  not  only  to  detect  anomalies,  but  also  to  diagnose  them,  gain  understanding of  their  underlying  causes,  and  ultimately  to  identify  mitigations.   We  plan to  adopt  a systematic  safety-analysis  and  assurance  approach  using  Fault  Tree  Analysis  (FTA)  and Safety-Assurance Cases (SAC). {\bf No progress was been made on this goal in August.}
\subsection{ Architectural Analysis and Integration: }
A key component of this research involves evaluating existing NAS and SWIM architecturesto determine how well they can support in-time assessment functions – such as our Dronolytics  solution.  We  will  design,  implement,  and  deploy  a  highly  modular,  scalable, low-power consumption prototype system with capabilities to interface with existing NAS infrastructures. 

\section{Questions from 6 Month Review Presentation}
\label{sec:response}

\begin{enumerate}
    \item {\bf [Steven Young] On one chart they talked of 63 flight logs used for looking at Roll, Vibration, and Compass anomalies. How did they select the 63 from the 53,000? Would it be difficult to look at more of the logs for these types of issues? To look for other issues?} \vspace{4pt} \newline To clarify, the 63 flight logs used in our experiments were $ArduPilot$ logs. Flightlogs in the train and validation datasets were taken from our own UAVs, whilst flight logs in the test set were all taken from open discussion forums in which the actual anomalies were broadly discussed and consensus about the anomaly was achieved. We used these flight logs for our initial study, primarily because we could confirm whether accidents or anomalous behavior had actually occurred, and in the case of the forum logs, had confirmation from experts.  As explained earlier in this report, our plan is to extend to the larger px4 dataset as already described in Task 1. \newline
    \noindent\fbox{%
    \parbox{0.93\columnwidth}{%
    {\bf Note: Copied from Task 1 above:}
    (1) Build functioning anomaly detectors for px4 and Ardupilot based on a small annotated dataset, (2) Once we achieve `decent' accuracy for each type of anomaly, then use the trained models to detect anomalies in the larger dataset and automatically annotate them with $candidate$ anomalies, (3) Use our annotation tool to evaluate the candidate anomalies.  For step $\#3$ we will have multiple trained evaluators look at each log.  As our current analysis is on Ardupilot logs, the next step (in January) is to repeat the same experiments against px4 logs -- again using flight logs discussed in the forums as the test set. Once we have affirmed and/or retrained the anomaly detector for px4 logs, we will commence with step $\#3$ in order to ultimately deliver a much larger dataset of annotated anomalies.  We will use some orthogonal threshold-based heuristic techniques to inspect all suspect parts of the flight log in order to partially address the recall problem.
    }}
\item {\bf [Steven Young] There was a chart with a Table of "common failures". The list looked good. The one item I didn't see was Lost Link (C2 or TM link). Did this condition occur in any of the logs? Or was it not part of the logged data?} \vspace{4pt} \newline
Good question. It wasn't part of the `default data' collected in the logs. Following a very quick inspection, I (JCH) don't see it as a monitorable property (but may be searching for the wrong term here: \url{https://dev.px4.io/v1.10\_noredirect/en/middleware/uorb\_graph.html}. Nafee can check for this more thoroughly. However, it is easily captured with a simple runtime monitor checking for heartbeats.

One issue we are working on with our own system is that our `Ground Control Station' (i.e., all our code for controlling the drone) is now actually onboard the drone. Therefore the meaning of the term $loss~of~signal$ actually changes as the primary communication signal comes directly from the onboard computer and the flight controller. We are instituting a heartbeat mechanisms between the onboard compute and the ground. 

Anyway, we can capture a very diverse set of attributes (far more than represented by the default log data), and can also implement a custom runtime monitor to capture additional data.

\item {\bf [Steven Young] On the detectors work, I'm curious how these differ from the COTS software. For example, the PX-4 software will flag excessive vibration and warn the user. Maybe the ND detectors are meant to be independent detectors in case the COTS fails or has excessive Pfa or Pmd. } \vspace{4pt} \newline
Currently, software systems deployed on/as ground control systems, such as QGroundControl (px4,Ardupilot) or MissionPlanner (Ardupilot) receive telemetry data from the flight controller. This includes data needed to identify vibration. In my experience flying drones, they have not provided this warning even though high vibration has occurred. Further, the post-flight analysis of flight logs can generate vibration graphs and interpret them; however, the results are visual and/or based on threshold values. In our IPSN paper we did compare threshold-based approaches with the deep learning models (ANN and LSTM) and found far more accurate results with a trained model.  Most importantly, (1) we need to acquire this data during flight in close to real-time, so that we can actually react to emergent problems, and (2) we need to untangle multiple symptoms -- which at this point we suspect occur together.  For example -- vibration might have several causes, so we want to look at multiple attributes in an effort to diagnose the problem and take correct migitations.


\item {\bf [Steven Young] On the performance analysis of the detectors, how is ``accuracy" calculated? Also, could they produce estimates for probability of false alarm (Pfa) and probability of missed detection (Pmd)? The latter is also used sometimes as a measure of integrity (I=1-Pmd).} \vspace{3pt}\newline We currently computed accuracy based on ground truth i.e., we know from the expert discussions in the forum whether the flight log includes one of our three targeted anomalies. We compute accuracy based on TP (True Positive), FP (False Positive), TN (True Negative), and FN (False Negative) -- computed for each flight logs as follows, where Accuracy = (TP+TN)/(TP+TN+FP+FN). Please see Algorithm 1 on the following page.

\begin{algorithm}
\SetAlgoLined
\For{each flight log in the test set}{
   \uIf{Ground truth indicates that an anomaly exists in the log}{
    \uIf{Anomaly is detected at the correct time period}{
      True Positive\;}
    \uElseIf{No anomaly is detected}{
      False Negative\;}
    \uElseIf{Anomaly is detected at incorrect time period}{
      False Positive\;}
   }\Else{
   \uIf{Anomaly is detected anywhere in the log}{
      False Positive\;}
   \Else{True Negative\;}}
    % Parse the log using the anomaly detector\;
    % \If{}{the anomaly exists in the log\;
    % \uif{}{the detector detects the anomaly at the correct location in the log file}
    % \uif{}{the detector detects the anomaly at the correct location in the log file}
    % }
} 
We hesitated to provide more complex metrics given our small dataset; however, we can certainly do so once we move to the larger curated dataset.

 \caption{Accuracy Computation for specific anomaly type}
 \label{pattern:1}
\end{algorithm}


\item {\bf [Steven Young] Curious about the SME's that were used. How many? Just opinion of one, or a consensus-based group. }  \vspace{4pt} \newline
For the training and validation datasets, it was the opinion of one. However, for the test dataset (on which results are based), the results were based on forum discussion and feedback plus our own confirmation of the problem.  We carefully read the entire discussion and looked for consensus decisions. 

\item {\bf [Steven Young] Anomaly detection work should know about Nikunj' body of work and recent work. Maybe something they can leverage or apply. On flip side, maybe Nikunj can leverage or apply some of the ND stuff. I wonder how difficult to give Nikunj access to the 53,000 logs, or a subset of them? Would it be worthwhile to pursue?} \vspace{4pt}\newline
Nitesh Chawla is very familiar with Nikunj' general body of work, but we hadn't previously seen his work on anomaly detection.  Thanks for pointing it out to us.  We will read all relevant papers we find. From Nikunj's google scholar profile, the most relevant paper we have found is: \vspace{3pt}\newline
Nikunj Oza, Kevin Bradner, David L. Iverson, Adwait Sahasrabhojanee and Shawn R. Wolfe. ``Anomaly Detection, Active Learning, Precursor Identification, and Human Knowledge for Autonomous System Safety'', AIAA 2021-1771. AIAA Scitech 2021 Forum. January 2021. \vspace{3pt}\newline
If there are any other particularly relevant papers, please let us know. 
Furthermore, we welcome Nikunj to collaborate with us and our students. 

\item {\bf [Additional question from presentation] To what extent are your results generalizable across different UAVs and environmental conditions?} \vspace{3pt}\newline This is still an open question; however, the Ardupilot experiments were conducted using three fairly diverse types of drones -- namely the America Spreading Wings
S900 hexcopter, and two smaller quadcopters (i.e., the 3DR Iris and the AeroHawk). We trained a shared model for each targeted anomaly using both ANN and LSTM, and observed that the models worked well on flight logs from all three types of UAVs. However, this question certainly warrants further investigation.

In addition, we have previously downloaded very detailed local weather conditions for all of our own Ardupilot flight logs. We haven't explored this data yet; however, we plan to do so. Based on our own experience, it is evident that certain flight anomalies are caused directly by low temperatures and/or high winds, and this will be an interesting experiment to explore in the future. (Note: once we setup our infrastructure, these kinds of additional experiments are great projects for our Undergraduate students looking to engage in research projects). 
\end{enumerate}

\section{Questions to NASA and other updates: }
Could we skip the January report and merge our December/January activities and results into the early February report?  The reason for this request is that universities will close in two weeks, so we would only be reporting on two weeks worth of work.  Some of our students are also taking their annual vacation time before the holidays. We are happy to write a short report if required, but would prefer to focus our limited working time on the SEAMS paper due January 20th, and submit our next report by February 7th.  Please let us know.
\end{document}



