\section{Goal 5: Architectural Analysis and Integration}
\vspace{-8pt}
\noindent{\it A key component of this research involves evaluating existing NAS and SWIM architectures to determine how well they can support in-time assessment functions -- such as our Dronolytics solution. We will design, implement, and deploy a highly modular, scalable, low-power consumption prototype system with capabilities to interface with existing NAS infrastructures. (Lead Investigators: Cleland-Huang, Vierhauser)}

\subsection{Analysis of NAS and SWIM Architectures}
The NASA document on ``Architecture and Information Requirements
to Assess and Predict Flight Safety Risks During
Highly Autonomous Urban Flight Operations'' \cite{nas-arch} describes the NAS, SWIM, and other architectural infrastructures. We will assess the ability of these architectures to support in-time monitoring, assessment, and mitigation. \cite{DBLP:journals/cacm/YoungL14}, utilizing the  Architectural Tradeoff Assessment Method (ATAM) \cite{KazmanATAMMethod2000,706657}, which we have used effectively on previous large projects . The ATAM process evaluates an architecture with respect to a set of desired qualities and we will take the qualities listed by Young et al., as a starting point \cite{nas-arch} (e.g.,scalability, extensibility, interoperability, data integrity assurance, security, privacy, fault-tolerance, and performance). Our analysis will involve establishing realistic performance benchmarks, data size, number of sUAS predicted for the next 10-25 years, and estimated flight and usage patterns (e.g., expected locations and flight durations). We will document our analysis in a detailed white paper.

\subsection{Architectural Design of Dronolytics}
We will leverage our existing and proven architectural infrastructure to support the data collection phase of the project (primarily years 1 and 2). We will then design, implement, and deploy the novel Dronolytics solution with  onboard and cloud-based capabilities and integration with services provided by the NAS as depicted in Figure \ref{fig:overview}. During the design phase we will review the previous ATAM analysis with respect to architectural support for Dronolytics. 
In our initial plans, the onboard analytics component -- deployed either as a Virtual Machine (VM) or in a Docker container, will interface with the sUAS' firmware using either the existing core Flight System (cSF) if available, or our existing code-base via supporting libraries (e.g., DroneCode, DroneKit Python) and ROS to communicate with diverse external sensors. It will communicate with the cloud-based component over telemetry, LTE, or wifi. Further, we will design the architecture to support sUAS with varying on-board compute capabilities including (i) high-performance (e.g., Jetson), (ii) basic capabilities (e.g., Raspberry Pi or limited proprietary compute), and finally (iii) sUAS without on-board compute. To facilitate these different approaches we will design Dronolytics using a modular approach consisting of (i) basic anomaly detection, (ii) simple diagnostics, and (iii) advanced diagnostics; and will allow none, some, or all of these capabilities to execute on the ground or on-board. When no on-board capabilities exist, all sensor data will be retrieved via telemetry or wifi connected directly to the autopilot from a control station (either on the ground or in the cloud). From the very start of the project we will design and implement integration points with the services provided by SWIM and NAS architectures. 

\subsection{Limitations and Risks}
\label{sec:architectureExtras}
\noindent\textbf{Sources of Error and Uncertainties.}
We have already integrated our current system with LAANCS, and will work to integrate with other services (e.g., AIXM, WXXM, and FIXM) in order to retrieve weather, navigation, and aeronautical information. As per the CFP we expect to meet with NASA engineers at the start of the project to discuss integration points and to explore the feasibility of a new service to support sUAS. %Our architectural analysis will be based on estimates; however, we will seek to estimate upper bounds, and will design scalable, service-based solutions.

\noindent\textbf{The Resilience of the Approach and Methodology.} All FTAs and SACs will be released into the public domain and will be open for discussion, debate, and further experimentation.

\noindent\textbf{Special capabilities and advantages of facilities and equipment.} As explained earlier in the proposal our existing Dronology system is built in a modular, scalable, and service-oriented way and already interfaces with AirMap (LAANC). We will leverage our expertise in the planned work.

\subsection{Technology Transfer}
\begin{wrapfigure}{r}{0.30\textwidth} 
\vspace{-10pt}
\centering
\includegraphics[width=.30\columnwidth]{figures/TechTransferFlights.jpg}
%\includegraphics[width=.38\columnwidth]{figures/hex.png}
\caption{Field tests with Dronology}
\label{fig:pm}
\end{wrapfigure} 
Our technology transfer plan is interwoven throughout the entire timeline. In Goal 1 we immediately start collecting real-world data so that we can ensure that our algorithms are effective for diverse real-world scenarios. Furthermore, this infrastructure can ultimately be integrated as an optional add-on for post-flight uploads to the proposed LAXM (see supplemental data service provider in Figure \ref{fig:overview}) to enhance descriptions of incidents with labeled data and contribute towards the goal of fully automated trend analysis. In Goals 2 and 3, we will not only evaluate algorithmic solutions for accuracy but also for response-time and throughput at high scale and with  low-energy consumption for any components targeted for on-board analysis. Goal 4 is inherently designed to support technology transfer as it analyzes our findings and solutions for safety and exposes the analysis and safety arguments for further scrutinization. Goal 5 ensures that the Dronolytics solution will be closely integrated into the current and proposed architecture -- with any stress factors impacting scale and performance clearly identified.

Finally, in year 3 we plan a series of integrated field-tests at a NASA center. The tests will include single and multi-sUAS scenarios using diverse sUAS. In these tests we plan to  minimally demonstrate that (1) Dronolytics is able to upload flight routes, retrieve flight authorizations with constraints, and integrate this data source into the onboard analytics, (2) use Dronolytics to monitor and assess flight data in `real-time' and detect anomalies as they occur, (3) transmit data in real-time to cloud-based Dronolytics services for further analysis and provide fast trend-detection of multi-sUAS emergent problems, (4) trigger notification to the NAS infrastructure for high-risk anomalies, and finally that (5) data can be automatically offloaded post-flight.  We will design rich test scenarios based upon combinations of the controlled test flights we used and evaluated in Goal 1. These test cases are inherently designed to trigger anomalous behavior which will enable more holistic field-tests throughout our project and during the technology transfer demonstrations. \textbf{All data and solutions developed through this work will be released to NASA for their use.}

