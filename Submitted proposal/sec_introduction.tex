\section{Introduction}
\label{sec:Intro}
A 2018 National Academies Report has outlined a vision of an 
\emph{In-time Safety Assurance Management System} (IASMS) that ``continuously monitors the national airspace system, assesses the data that it has collected, and then either recommends or initiates safety assurance actions as necessary'' \cite{NAP24962}. The report highlights the need for on-board, close to real-time assessment, augmented by off-board trend analysis. We propose a solution that supports this vision through in-time safety assurance of Small Unmanned Aerial Systems (sUAS). Both the number of sUAS deployed in the national air space (NAS), and the number of accidents and/or unauthorized infringements into controlled airspace have risen dramatically in recent years, increasing the risk of serious accidents ~\cite{RISK1,RISK2}.  %We recently analyzed over 290 publicly available incident reports and identified hardware defects, system or UI flaws, and deliberate or accidental pilot error as the primary cause of accidents. Many incidents involved human initiated error such as flying BVLOS (beyond visual line of sight), failing to obtain permission to fly in controlled airspace, flying in unsafe proximity to traffic or people on the ground, or incorrect firmware configurations. However, these formal incident reports do not include numerous additional incidents that fall below the Federal Aviation Authority's (FAA) reporting requirements.  Such incidents include critical system failures such as loss of link, loss of geolocation, loss of power, motor failures, and flight control system failures many of which lead to complete loss-of-control.
\begin{wrapfigure}[17]{r}{0.45\textwidth} 
\vspace{-12pt}
\centering
%\setlength{\lineskip}{\medskipamount}
\subcaptionbox{The sUAS actual flight path. \label{fig:1a}}{\centering\includegraphics[width=.42\columnwidth]{figures/Accident1e.PNG}}
\subcaptionbox{The sUAS failed to respond to commands. \label{fig:1b}}{\centering\includegraphics[width=0.42\columnwidth]{figures/Accident1b.png}}
\vspace{-4pt}
\caption{Flight logs of an \emph{out-of-control} sUAS}
\label{fig:HexCrash}
\end{wrapfigure}
In this proposal we primarily focus on failures and risks associated with  \emph{out-of-control} sUAS, potentially caused by envelope excursions, flight control system failures, and/or loss of signal. This is a situation which almost every remote pilot in command (RPIC) has experienced more than once. Figure \ref{fig:HexCrash} visualizes the flight logs of an incident with our own American HexCopter running Ardupilot. The sUAS took off in GUIDED mode (red) and was directed to fly to a waypoint immediately North West of its launch position. Instead, after an initial attempt to fly North, it headed South East. At that point, the system issued a LAND command (green) followed by a HOVER command (purple) both of which were ignored. The RPIC then attempted to take manual control of the sUAS, and again issued an RTL (blue) command followed by a LAND command (green) directly from the hand-held controller -- both of which were ignored and the sUAS ultimately crashed into a tree. We speculated that the incident was caused by an emergent magnetometer failure although this is hard to confirm without more sophisticated analytic tools such as the one we propose here.  

\begin{figure*}[!t]
    \centering
    \includegraphics[width=0.96\textwidth]{figures/Overview5.pdf}
    \caption{Dronolytics supports on-board and off-board risk assessment through predictive analytics. It leverages heterogeneous data from onboard sensors, human operators, and NAS Data Service providers to perform in-time on-board and off-board analytics, and sends safety notifications, alerts, and annotated flight data back to the NAS infrastructure.. Our five primary research foci are labeled 1-5.}
    \label{fig:overview}
    \vspace{-8pt}
\end{figure*}

\subsection{Research Goals}
Our Dronolytics solution combines a novel, and cutting-edge predictive analytics approach, with combinatorial integration testing to deliver on-board and off-board assessment of emergent sUAS safety risks. Our approach detects both seen and unseen flight anomalies,  identifies risky feature interactions, and augments the training process with crowd-sourced explanations of the failure events. Dronolytics, summarized in Figure \ref{fig:overview}, contributes towards the In-time Aviation Safety Management System (IASMS) by providing timely notification of sUAS safety risks as they occur, with a particular emphasis on  out-of-control sUAS.  Our work directly addresses Software Safety Project B5.2.2 (Subtopic 2) to develop ``novel on-line and Automated Analytics-Based Risk Assessment with Predictive Capability'' through the following research goals. 

\begin{itemize}[leftmargin=-0.1em]
    \setlength\itemsep{.2em}
    \item[] {\bf Goal 1: Develop a diverse data set of annotated sUAS flight logs.} To explore innovative solutions for in-time detection and diagnosis of sUAS faults we urgently need a publicly available, diverse, data set of flight logs. Our first research goal therefore takes a multi-prong approach to systematically collect annotated, augmented physical and simulated flight-logs for both normal and abnormal flights. ({\it Cleland-Huang})
    \item[] {\bf Goal 2: Detect and Interpret Anomalous Flight Patterns.} This research goal focuses on differentiating normal from abnormal patterns of flight-associated data. We propose innovative algorithms for dealing with heterogeneous, multivariate, potentially noisy data sets in order to mine expected patterns of diverse flight data, detect aberrations, and classify them in order to support mitigations and alerts. {\it (Chawla)}
    
    \item[] {\bf Goal 3: Discover risky feature interactions.}  Unexpected emergent system behaviors often occur due to feature and environmental interactions which may be hard to detect using predictive analytics without extremely large and varied datasets of anomalies. We therefore systematically explore these interactions using combinatorial interaction testing and a search-based  approach to a priori identify risky interactions that could lead to loss of control. Results are integrated as tagged data into the predictive analytics approach in order to deliver a highly-novel multi-faceted solution. {\it (Cohen)}
  
    \item[] {\bf Goal 4: Safety Analysis and Failure Mitigation.} Our primary aim is to assess and respond to safety risks in sUAS. In this research goal we focus on the safety analysis of Dronolytics monitoring, assessment, and mitigations.  {\it (Cleland-Huang, Cohen)} 
    
    \item[] {\bf Goal 5: Architectural Analysis, Integration, and Technology Transfer } We perform a critical architectural analysis of National Airspace (NAS), Air Navigation Service (AINS), SWIM, and other relevant architectures to assess its support for sUAS monitoring, assessment, and mitigation services such as the Dronolytics solution. {\it ({Cleland-Huang, Vierhauser (Unfunded Collaborator)})}
\end{itemize}

\subsection{Team}
Our team is well equipped for this research project bringing expertise with sUAS, safety-assurance, and runtime monitoring (Jane Cleland-Huang), data science (Nitesh Chawla), and feature interactions with combinatorial testing (Myra Cohen). Cleland-Huang has worked extensively in the sUAS space over the past six years with relevant work in human-drone interactions \cite{ankit-chi2020, ClelandHuang-iHDI2020}, sUAS product lines \cite{SPLC}, runtime monitoring \cite{DBLP:conf/euromicro/VierhauserCBKRG18}, and safety-assurance for sUAS \cite{tse-preprint,AgileTraceability,DBLP:conf/icse/AgrawalKVRCL19}. Chawla is a renowned expert in data science with over 33,000 citations and h-index of 62 (source Google Scholar) and highly innovative recent work in the area of (multi-variate) time-series modeling and anomaly detection ~\cite{saebi2020efficient, lin2020filling, huang2019deep, huang2019mist, zhang2019deep, wu2018restful, wu2019neural, wu2020hierarchically}.  Cohen has extensive expertise in testing of configurable software \cite{YuSCR18,DBLP:conf/kbse/CashmanCRC18,Shi:2008,Qu:2008,JinQCR14,garvin:issre}, combinatorial interaction testing \cite{masking:2014,Shi:2008,PetkeCHY15}, adverse feature interactions in diverse cyber-physical systems \cite{DBLP:conf/sigsoft/MansoorSSBCF18,DBLP:conf/kbse/CashmanCRC18} and search-based software engineering\cite{YuSCR18,JiaCHP15}. Our external  collaborator Michael Vierhauser (Johannes Kepler University Linz, Austria) has expertise in runtime monitoring of Cyber-Physical Systems \cite{vierhauser2016reminds,vierhauser2016requirements,tse-preprint} and is co-architect of the Dronology system.%, while Bertrand Hochwald (Co-Director of the Wireless Institute, University of Notre Dame), a 2019 National Academy of Inventors Feller who brings expertise in radio spectrum monitoring \cite{radiohound} provides use of RadioHound for in-house flight logging. (Please see letters).



